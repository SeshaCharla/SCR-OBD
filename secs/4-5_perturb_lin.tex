\subsection{Perturbation Models}
The perturbation model would bring down the change in
temperature from exponent to the algebriac form. We have the first-order tayler
approximation of rate constant:
\begin{align*}
    k(\bar T + \delta T) &\approx k(\bar T) + \frac{E_a}{R\bar T^2} k(\bar T) \delta T\\
    \delta k &\approx p \bar k \delta T\\
    \text{Where,} \quad &\\
    p &= \frac{E_a}{RT_0^2}
\end{align*}
Introducing the above approximation into the lumped parameters from eqn.\ref{eqn::full_nonlinear}:
\begin{align*}
    f_{13} &= k_1 \quad&
    \implies \bar f_{13} &= \bar k_{1}, \quad&
    \delta f_{13} &= p_{1} \bar k_{1} \delta T\\
    %===
    f_{23} &= k_{4F} \quad &
    \implies \bar f_{23} &= \bar k_{4F}, \quad &
    \delta f_{23} &= p_{4F} \bar k_{4F} \delta T\\
    %===
    f_{32} &= k_{4F} \Theta \quad &
    \implies \bar f_{32} &= \bar k_{4F} \bar \Theta, \quad &
    \delta f_{32} &= p_{32}  \bar k_{32} \bar \Theta \delta T   +  \bar k_{4F} \delta \Theta
    \\
    f_{31} &= k_1 V \quad &
    \implies \bar f_{31} &= \bar k_1 V \quad &
    \delta f_{31} &= p_1 \bar k_1 V \delta T
    \\
    f_{24} &= b_v F
    \quad &
    \implies \bar f_{24} &= b_v \bar F
    \quad &
    \delta f_{24} &= b_v \delta F
    \\
    g_1    &= b_v F
    \quad &
    \implies \bar g_1 &= b_v \bar F
    \quad &
    \delta g_1 &= b_v \delta F
    \\
    g_2    &= b_v F + k_{4F} \Theta
    \quad &
    \implies \bar g_2 &= b_v \bar F + \bar k_{4F} \bar \Theta
    \quad &
    \delta g_2 &= b_v \delta F +  p_{32}  \bar k_{32} \bar \Theta \delta T   +  \bar k_{4F} \delta \Theta
    \\
    g_{3}  &= k_{4R}+k_3
    \quad &
    \implies \bar g_3 &= \bar k_{4R} + \bar k_3
    \quad &
    \delta g_3 &= \lr{ p_{4R} \bar k_{4R} + p_3 \bar k_3} \delta T
    \\
    g_{23} &= k_{4R} b_v
    \quad &
    \implies \bar g_{23} &= \bar k_{4R} b_v
    \quad &
    \delta g_{23} &= p_{4R} \bar k_{4R} b_v  \delta T
    \\
    g_{32} &= k_{4F} V \Theta
    \quad &
    \implies \bar g_{32} &= \bar k_{4F} b_v
    \quad &
    \delta g_{32} &= p_{4F} \bar k_{4F} b_v  \delta T
    \\
    g_4 &= \omega_u
    \\
    b_{11} &= b_v F
    \quad &
    \implies \bar b_{11} &= b_v \bar F
    \quad &
    \delta b_{11} &= b_v \delta F
    \\
    b_{42} &= \frac{\omega_u b_u}{F}
    \quad &
    \implies \bar b_{42} &= \frac{\omega_u b_u}{\bar F}
    \quad &
    \delta b_{42} &= -\frac{\omega_u b_u}{\bar F^2} \delta F
\end{align*}
