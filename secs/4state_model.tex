\subsection{4 State Dynamic Model}
A four state nonlinear model for the above reactions can be developed using
Arrhenius equations, CSTR assumption and further simplification based on the
following assumptions:

\subsubsection{Assumptions}
\begin{enumerate}
    \item Slow SCR reaction is neglected.
    \begin{itemize}
        \item The flow rate of the exhaust would ensure that the not a significant concentration of tail-pipe exhaust components are due to the slow SCR reaction \cite{nova2014urea}.
    \end{itemize}

    \item Mass transfer is neglected. That means the chemical kinetics in the catalyst are reaction controlled.
    \begin{itemize}
        \item The standard SCR reaction rate is faster than the flow rate of the exhaust fluids.
    \end{itemize}

    \item Nitrogen selectivity for ammonia oxidation is $100\%$.
    \begin{itemize}
        \item This assumption is relaxed by including algebraic relationship between selectivity and the temperature (ASC model \cite{jain2023diagnostics}).
    \end{itemize}

    \item Reaction rates are assumed to be a function of the gas phase concentration of $NO_x$ and ammonia storage.
\end{enumerate}


\subsubsection{Reaction Rates}
The reaction rates for the processes being considered are modelled using
Arrhenius equations as follows:
\begin{enumerate}
\item Standard SCR Reaction (\ref{eqn::std_scr}):
$ R_1 = \underbrace{k_1 \exp\lr{-\frac{E_1}{RT}}}_{r_1} C_{NO} C_{O_2} \theta \Theta V^2 $

\item Fast SCR Reaction (\ref{eqn::fast_scr}):
$ R_2 = k_2 \exp\lr{-\frac{E_2}{RT}} C_{NO} C_{N_2O} \theta \Theta V^2 $

\item Ammonia Oxidation (\ref{eqn::amox_N2}):
$ R_3 = \underbrace{k_3 \exp\lr{-\frac{E_3}{RT}}}_{r_3} C_{O_2} \theta \Theta
V $

\item Ammonia Adsorption/Desorption (\ref{eqn::ads}):
\begin{enumerate}
    \item Forward: $ R_{4F} = \underbrace{k_{4F} \exp \lr{-\frac{E_{4F}}{RT}}}_{r_{4F}} C_{NH_3} \lr{1-\theta} \Theta V$

    \item Reverse: $ R_{4R} = \underbrace{k_{4R} \exp \lr{-\frac{E_{4R}}{RT}}}_{r_{4R}}
\Theta V $
\end{enumerate}

\item NO oxidation (\ref{eqn::NOX}):
$ R_5 = \underbrace{k_5 \exp \lr{- \frac{E_5}{RT}}}_{r_5} C_{NO} C_{O_2} V^2 $
\end{enumerate}

Where,
\begin{align*}
    \theta &- NH_3 \text{ storage capacity fraction in SCR } = \frac{\text{Moles of $NH_3$ adsorbed}}{\text{Total moles of $NH_3$ that can be adsorbed}}\\
    \Theta &- \text{Ammonia storage capacity} (moles)\\
    \Theta &= S_1 e^{S_2 T} \qquad \qquad \begin{matrix*}[l]
                S_1, S_2 &-& \text{Aging parameters of the catalyst (positve constants)}
            \end{matrix*}\\
    E_i &- \text{Activation Energy of $i^{th}$ reaction}\\
    k_i &- \text{Pre-exponential factor}\\
    R &- \text{Universal gas constant}\\
    T &- \text{Temperature}\\
    C_{\{\bullet\}} &- \text{Concentration} \lr{mol/m^3}\\
    V &- \text{Volume of the exhaust gas in the substrate\cite{devarakonda2009model}} \lr{m^3}\\
    V_e &= \epsilon A_c L_{cat}\\
        &\begin{matrix*}[l]
        A_c &-& \text{Open frontal area of the catalyst}\\
        L_{cat} &-& \text{Length of the catalyst}\\
        \epsilon &-& \text{Void fraction}
        \end{matrix*}\\
\end{align*}

\subsubsection{Dynamic model}
Using above assumptions and definitions, we have the dynamic model \cite{nova2014urea}:
\begin{equation} \label{eqn::4_state}
    \bm{\dot C_{NO} \\
        \dot C_{NO_2}\\
        \dot C_{NH_3}\\
        \dot \theta_{NH_3}\\
        } =
    \bm{
        -r_1 C_{NO} C_{O_2} \theta_{NH_3} \Theta V
        - 0.5 r_2 C_{NO} C_{NO_2} \theta_{NH_3} \Theta V
        -r_5 C_{NO} C_{O_2} V
        -b C_{NO}\\
        %===
        -0.5 r_2 C_{NO} C_{NO_2} \theta_{NH_3} \Theta V
        + r_5 C_{NO} C_{O_2} V
        -b C_{NO_2}\\
        %===
        -C_{NH_3} \lrb{\Theta r_{rF} \lr{1 - \theta_{NH_3}} + b} + V^{-1} r_{4R} \Theta \theta_{NH_3}\\
        %===
        -\theta_{NH_3} \lr{r_{4F}C_{NH_3} V + r_3 C_{O_2}V + r_{4R} + r_1 C_{NO} C_{O_2} V^2 + r_2 C_{NO} C_{NO_2} V^2} + r_{4F} C_{NH_3} V
    }
    + b \bm{C_{NO, in} \\ C_{NO_2, in} \\ C_{NH_3} \\ 0}
\end{equation}
Where,
\begin{align*}
    b = \frac{F}{V}
\end{align*}
