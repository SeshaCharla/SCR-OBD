\section{Linearized Model Identification}
The linearized model derived previously can be used for model parameter
identification. The idea is to identify the lumped linear model parameters and
then use this information to arrive at the estimates for the relevant parameters
for prediction/aging-factor estimation purposes. The following are assumed
apriori :

\begin{enumerate}
\item Few of the lumped parameters of the linearized model are enough for
obtaining the necessary model parameters.
\item These lumped parameters can be estimated using the data.
\end{enumerate}

\subsection{Transfer function estimation approach}
Since we have uncorrupted measurement data for $NO_x$ and $NH_3$ and all the
inputs at any given time, part of the state-to-input transfer function matrix
can be estimated using this data. The state-space model form
section-\ref{eqn::lin_model} into frequency domain, the model structure would
be:
\begin{align*}
    \bm{\delta x_1 \\ \delta x_2 \\ \hline \delta  x_3 \\ \delta x_4} &= \bm{G_{11} & G_{12} & G_{13} & G_{14}\\
                G_{21} & G_{22} & G_{23} & G_{24}\\
                \hline
                G_{31} & G_{32} & G_{33} & G_{34}\\
                G_{41} & G_{42} & G_{43} & G_{44}\\
    }
    \bm{\delta u_1 \\ \delta u_2 \\ \delta T \\ \delta F}
\end{align*}

With the given data, $G_{1i}'s, G_{2i}'s$ can be estimated. The presumption of
this approach is that we should be able to get estimates of most of the required
parameters from these two rows.

\input{secs/8-1-1_TransferFuncModel.tex}
