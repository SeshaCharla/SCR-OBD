\section{Catalyst aging factor}

We have the ammonia storage capacity of the catalyst:
\begin{align*}
    \Theta &= S_1 e^{-S_2 T}
\end{align*}

The parameters $S_1$ and $S_2$ change with age effecting the storage capacity at
a given temperature.

\itbf{Assumption}: The aged catalyst results in small changes in
$S_1, S_2$.

The above assumption is valid if the catalyst's operating range is limited to a
small range of storage capacity.

Using small perturbation,
\begin{align*}
    \delta \Theta &= \lr{\frac{\delta S_1}{S_1} - \delta S_2T} S_1 e^{-S_2 T}\\
    \implies \Theta_{aged} &= \Theta + \delta \Theta = \lr{1 + \frac{\delta S_1}{S_1} - \delta S_2T} S_1 e^{-S_2 T}
\end{align*}

Let,
\begin{align*}
    a(T) &= 1 + \frac{\delta S_1}{S_1} - \delta S_2T = a_1 + a_2 T
\end{align*}

Thus, $a$ is the factor by which the storage capacity is reduced due to the
catalyst's aging. Hence, for optimal performance:
\begin{align*}
    a > a_{min} \quad \forall T \in [T_{min}, T_{max}]
\end{align*}

\itbf{Note}: The above definition is consistent with that of the literature
\cite{ma2017observer}. The major difference lies in its derivation and
assumptions. Also, \cite{ma2017observer} considers aging factor as temperature
independent fraction and has no minimum value for classifying the catalyst as aged.

Consequently, \itbf{the catalyst aging detection problem becomes estimating the
aging factor and testing if it is bellow $a_{min}$ in presence of
uncertainties}.
