\section{SCR-ASC Reactions and Dynamics}
The SCR reactions involve reduction of $NO_x$ into $N_2$ and $H_2 O$ \cite{jain2023diagnostics}.

\subsection{SCR-ASC Reactions}
Eley Rideal reaction mechanism \cite{hsieh2011development}
\cite{yuan2015diesel}, \cite{nova2014urea} is considered for the interpreting
the SCR reactions. The mechanism involves the following reactions:

\begin{align}
    NH_2 - CO - NH_2 (liquid) &\longrightarrow NH_2 - CO - NH_2^* + x H_2 O
                & &[\text{AdBlue evaporation}] \label{eqn::urea_1} \\
    NH_2 - CO - NH_2^*  &\longrightarrow  HNCO + NH_3
                & &[\text{Urea decomposition}] \label{eqn::urea_2}\\
    HNCO + H_2O &\longrightarrow NH_3 + CO_2
                & &[\text{Isocynic acid hydrolysis}] \label{eqn::urea_3}\\
    %===
    NH_3 + \theta_{free} &\longleftrightarrow NH_3(ads)
                & &[\text{Adsorption/Desorption}] \label{eqn::ads}\\
    %===
    4 NH_3 (ads) + 4 NO + O_2 &\longrightarrow 4 N_2 + 6 H_2O
                              & &[\text{Standard SCR reaction}]
                              \label{eqn::std_scr}\\
    %===
    2 NH_3 (ads) +  NO + N O_2 &\longrightarrow 2 N_2 + 3 H_2O
                              & &[\text{Fast SCR reaction}]
                              \label{eqn::fast_scr}\\
    %===
    4 NH_3 (ads) + 3N O_2 &\longrightarrow 3.5 N_2 + 6 H_2O
                              & &[\text{Slow SCR reaction}]
                              \label{eqn::slow_scr}\\
    %===
    4 NH_3 + 3 O_2 &\longrightarrow 2 N_2 + 6 H_2O
                         & &[\text{AMOX with/without ASC}]
                         \label{eqn::amox_N2}\\
    4 NH_3 + 5 O_2 &\longrightarrow 4 NO + 6 H_2 O
                         & &[\text{AMOX with/without ASC}]
                         \label{eqn::amox_NO}\\
    2 NH_3 + 2 O_2 &\longrightarrow N_2O + 3 H_2O
                         & &[\text{AMOX with/without ASC}]
                         \label{eqn::amox_N20}\\
    %==
    2 NO + O_2 &\longrightarrow 2 NO_2
                        & &[\text{NO oxidation}]
                        \label{eqn::NOX}
\end{align}


The PDE model for SCR reaction kinematics \cite{nova2014urea} is reduced to ODE model \cite{devarakonda2008adequacy} using by assuming "continuous stirred tank reactor (CSTR)" model (Control volume approach).

\subsection{Ammonia input (Urea Dosing) dynamics}
The actual input to the system is urea from AdBlue ($32.5\%$ aqueous urea
solution) injection that converted to ammonia (through reactions:
(\ref{eqn::urea_1}), (\ref{eqn::urea_2}) and (\ref{eqn::urea_3})). This can be modelled by the following eqation \cite{nova2014urea}:
\begin{align*}
    \dot C_{NH_3, in} &= - \frac{1}{\tau} C_{NH_3, in} + 2 \frac{1}{\tau} \frac{ \eta u_{Adblue}}{N_{urea} F}\\
    \text{where, } \quad &\\
    \tau &- \text{Time constant}\\
    u_{AdBlue} &- \text{Mass injection rate of the AdBlue solution}\\
    \eta &- \text{Mass fraction of urea in the solution}\\
    N_{urea} &- \text{Atomic number of urea}\\
    F &- \text{Exhaust flow rate of the catalyst } m^3/s
\end{align*}
The above model is re-parametrized as follows:
\begin{align*}
    \dot C_{NH_3, in} &= \frac{1}{\tau} \lr{ -C_{NH_3, in} +\eta_{urea} C_{NH_3, ideal}} \\
    \text{where, }\qquad &\\
    \eta_{urea} &- \text{urea} \rightarrow NH_3 \text{ conversion efficiency}\\
    \tau &- \text{urea} \rightarrow NH_3 \text{ time-constant}\\
    C_{NH_3, ideal} &- \text{ideal $C_{NH_3, in}$ if $\eta_{urea} = 1$ (constant parameter obtained through calibration)}
\end{align*}
In the present work, $C_{NH_3, in}$ is considered as the input instead of AdBlue
injection as its dynamics are completely decoupled from that of other states.
Further, it is observed that AdBlue is completely converted to Ammonia at the
very upstream part of the SCR catalyst \cite{hsieh2011development}.

\subsection{4 State Dynamic Model}
A four state nonlinear model for the above reactions can be developed using
Arrhenius equations, CSTR assumption and further simplification based on the
following assumptions:

\subsubsection{Assumptions}
\begin{enumerate}
    \item Slow SCR reaction is neglected.
    \begin{itemize}
        \item The flow rate of the exhaust would ensure that the not a significant concentration of tail-pipe exhaust components are due to the slow SCR reaction \cite{nova2014urea}.
    \end{itemize}

    \item Mass transfer is neglected. That means the chemical kinetics in the catalyst are reaction controlled.
    \begin{itemize}
        \item The standard SCR reaction rate is faster than the flow rate of the exhaust fluids.
    \end{itemize}

    \item Nitrogen selectivity for ammonia oxidation is $100\%$.
    \begin{itemize}
        \item This assumption is relaxed by including algebraic relationship between selectivity and the temperature (ASC model \cite{jain2023diagnostics}).
    \end{itemize}

    \item Reaction rates are assumed to be a function of the gas phase concentration of $NO_x$ and ammonia storage.
\end{enumerate}


\subsubsection{Reaction Rates}
The reaction rates for the processes being considered are modelled using
Arrhenius equations as follows:
\begin{enumerate}
\item Standard SCR Reaction (\ref{eqn::std_scr}):
$ R_1 = \underbrace{k_1 \exp\lr{-\frac{E_1}{RT}}}_{r_1} C_{NO} C_{O_2} \theta \Theta V^2 $

\item Fast SCR Reaction (\ref{eqn::fast_scr}):
$ R_2 = k_2 \exp\lr{-\frac{E_2}{RT}} C_{NO} C_{N_2O} \theta \Theta V^2 $

\item Ammonia Oxidation (\ref{eqn::amox_N2}):
$ R_3 = \underbrace{k_3 \exp\lr{-\frac{E_3}{RT}}}_{r_3} C_{O_2} \theta \Theta
V $

\item Ammonia Adsorption/Desorption (\ref{eqn::ads}):
\begin{enumerate}
    \item Forward: $ R_{4F} = \underbrace{k_{4F} \exp \lr{-\frac{E_{4F}}{RT}}}_{r_{4F}} C_{NH_3} \lr{1-\theta} \Theta V$

    \item Reverse: $ R_{4R} = \underbrace{k_{4R} \exp \lr{-\frac{E_{4R}}{RT}}}_{r_{4R}} \theta
\Theta V $
\end{enumerate}

\item NO oxidation (\ref{eqn::NOX}):
$ R_5 = \underbrace{k_5 \exp \lr{- \frac{E_5}{RT}}}_{r_5} C_{NO} C_{O_2} V^2 $
\end{enumerate}

Where,
\begin{align*}
    \theta &- NH_3 \text{ storage capacity fraction in SCR } = \frac{\text{Moles of $NH_3$ adsorbed}}{\text{Total moles of $NH_3$ that can be adsorbed}}\\
    \Theta &- \text{Ammonia storage capacity} (moles)\\
    \Theta &= S_1 e^{S_2 T} \qquad \qquad \begin{matrix*}[l]
                S_1, S_2 &-& \text{Aging parameters of the catalyst (positve constants)}
            \end{matrix*}\\
    E_i &- \text{Activation Energy of $i^{th}$ reaction}\\
    k_i &- \text{Pre-exponential factor}\\
    R &- \text{Universal gas constant}\\
    T &- \text{Temperature}\\
    C_{\{\bullet\}} &- \text{Concentration} \lr{mol/m^3}\\
    V &- \text{Volume of the exhaust gas in the substrate\cite{devarakonda2009model}} \lr{m^3}\\
    V_e &= \epsilon A_c L_{cat}\\
        &\begin{matrix*}[l]
        A_c &-& \text{Open frontal area of the catalyst}\\
        L_{cat} &-& \text{Length of the catalyst}\\
        \epsilon &-& \text{efficiency factor}
        \end{matrix*}\\
\end{align*}

\subsubsection{Dynamic model}
Using above assumptions and definitions, we have the dynamic model \cite{nova2014urea}:
\begin{align*}
    \bm{\dot C_{NO} \\
        \dot C_{NO_2}\\
        \dot C_{NH_3}\\
        \dot \theta_{NH_3}\\
        } &=
    \bm{
        -r_1 C_{NO} C_{O_2} \theta_{NH_3} \Theta V
        - 0.5 r_2 C_{NO} C_{NO_2} \theta_{NH_3} \Theta V
        -r_5 C_{NO} C_{O_2} V
        -s_v C_{NO}\\
        %===
        -0.5 r_2 C_{NO} C_{NO_2} \theta_{NH_3} \Theta V
        + r_5 C_{NO} C_{O_2} V
        -s_v C_{NO_2}\\
        %===
        -C_{NH_3} \lrb{\Theta r_{rF} \lr{1 - \theta_{NH_3}} + s_v} + V^{-1} r_{4R} \Theta \theta_{NH_3}\\
        %===
    }
    + s_v \bm{C_{NO, in} \\ C_{NO_2, in} \\ C_{NH_3} \\ 0}
\end{align*}

\subsection{3 state dynamic model}
\subsubsection{Assumptions }
The following are the additional assumptions along with the assumptions of
4-state model that are used to arrive at the three-state dynamic model\cite{devarakonda2009model}:
\begin{enumerate}
    \item Only the standard SCR reaction is considered.
    \item All $NO_x$ in the exhaust gas is assumed to be $NO$.
    \begin{itemize}
        \item The commercially available $NO_x$ sensor (Horiba gas analyzer \cite{nova2014urea}) cannot differentiate between $NO$ and $NO_2$.
    \end{itemize}
\end{enumerate}
The above two assumptions result in the following:
\begin{align*}
    C_{NO_2} &= 0 \qquad C_{NO_2,  in} = 0\\
    r_5 &= 0
\end{align*}

\subsubsection{Dynamic Model}
Including the above assumptions in the 4-state dynamic model
(\ref{eqn::4_state}):

\begin{equation} \label{eqn::3_state}
    \bm{\dot C_{NO} \\
        \dot C_{NH_3}\\
        \dot \theta_{NH_3}\\
        } =
    \bm{
        -r_1 C_{NO} C_{O_2} \theta_{NH_3} \Theta V
        -b C_{NO}\\
        %===
        -C_{NH_3} \lrb{\Theta r_{rF} \lr{1 - \theta_{NH_3}} + b} + V^{-1} r_{4R} \Theta \theta_{NH_3}\\
        %===
        -\theta_{NH_3} \lr{r_{4F}C_{NH_3} V + r_3 C_{O_2}V + r_{4R} + r_1 C_{NO} C_{O_2} V^2 } + r_{4F} C_{NH_3} V
    }
    + b \bm{C_{NO, in} \\ C_{NH_3, in} \\ 0}
\end{equation}

The following parameters are defined for convenience:
\begin{align*}
    \mat{
    p_1 = r_1 C_{O_2} V
    &p_2 =  \frac{r_{4R}}{V}
    &p_3 = r_{4F} V
    &p_4 = r_3 C_{O_2} V
    &p_5 = r_1 C_{O_2} V^2
    }
\end{align*}
\begin{align*}
     \bm{\dot C_{NO} \\
        \dot C_{NH_3}\\
        \dot \theta_{NH_3}\\
        } &=
    \bm{
        -p_1 C_{NO} \theta_{NH_3} \Theta
        -b C_{NO}
        \\
        %===
        -C_{NH_3} \lr{\Theta r_{4F} + b}
        + r_{4F} \Theta C_{NH_3} \theta_{NH_3}
        + p_2 \Theta \theta_{NH_3}
        \\
        %===
        -p_3C_{NH_3} \theta_{NH_3}
        -\lr{p_4 + r_{4R}} \theta_{NH_3}
        -p_5 C_{NO} \theta_{NH_3}
        + p_3 C_{NH_3}
    }
    + b \bm{C_{NO, in} \\ C_{NH_3, in} \\ 0}
\end{align*}
Defining the following coefficients of product of states and states:
\begin{align*}
    \text{Coefficients of product of states:} &\qquad& \text{Coefficients of states:}\\
    \mat{             & C_{NO} & C_{NH_3} & \theta_{NH_3} \\
        C_{NO}        &        & f_{12}   &        \\
        C_{NH_3}      &        &          & f_{23} \\
        \theta_{NH_3} & f_{31} & f_{32}   &}
    &\qquad &
    \mat{             & C_{NO} & C_{NH_3} & \theta_{NH_3} \\
        C_{NO}        & g_1    &          &        \\
        C_{NH_3}      &        & g_{2}    & g_{23}        \\
        \theta_{NH_3} &        & g_{32}   & g_{3}}
\end{align*}
\begin{align*}
    \mat{
    \\f_{12} &=& p_1 \Theta    &=& r_1 C_{O_2} V \Theta
    \\f_{23} &=& r_{4F} \Theta & &
    \\f_{32} &=& p_3           &=& r_{4F} V
    \\f_{31} &=& p_5           &=& r_1 C_{O_2} V^2}
    \qquad \qquad
    \mat{
    \\ g_1    &=& b                      & &
    \\ g_2    &=& \lr{\Theta r_{4F} + b} & &
    \\ g_{3}  &=& p_4 + r_{4R}           &=& r_3 C_{O_2} V + r_{4R}
    \\ g_{23} &=& p_2 \Theta             &=&  \frac{r_{4R}}{V} \Theta
    \\ g_{32} &=& p_3                    &=& r_{4F} V
    }
\end{align*}
\begin{equation*}
     \bm{\dot C_{NO} \\
        \dot C_{NH_3}\\
        \dot \theta_{NH_3}\\
        } =
    \bm{
        -f_{12} C_{NO} \theta_{NH_3}
        -g_1 C_{NO}
        \\
        %===
        -g_2 C_{NH_3}
        + f_{23} C_{NH_3} \theta_{NH_3}
        + g_{23} \theta_{NH_3}
        \\
        %===
        -f_{32} C_{NH_3} \theta_{NH_3}
        -g_3 \theta_{NH_3}
        -f_{31} C_{NO} \theta_{NH_3}
        + g_{32} C_{NH_3}
    }
    + b \bm{C_{NO, in} \\ C_{NH_3, in} \\ 0}
\end{equation*}
Let,
\begin{align*}
    x_1 &= C_{NO} &  u_1 &= C_{NO, in}\\
    x_2 &= C_{NH_3} & u_2 &= C_{NH_3, in}\\
    x_3 &= \theta_{NH_3} & &
\end{align*}
\begin{equation}\label{eqn::parm_model}
     \bm{\dot x_1 \\
        \dot x_2\\
        \dot x_3\\
        } =
    \bm{
        -f_{12} x_1 x_3
        -g_1 x_1
        \\
        %===
        -g_2 x_2
        + f_{23} x_2 x_3
        + g_{23} x_3
        \\
        %===
        -f_{32} x_2 x_3
        -g_3 x_3
        -f_{31} x_1 x_3
        + g_{32} x_2
    }
    + b \bm{u_1 \\ u_2 \\ 0}
\end{equation}


\subsubsection{Small Perturbation model}
We have the small-perturbation model from eqn.~\ref{eqn::parm_model}:
\begin{equation}\label{eqn::sml_ptrb}
     \bm{\delta \dot x_1 \\
        \delta \dot x_2\\
        \delta \dot x_3\\
        } =
    \bm{
        -\lr{g_1 + f_{12} x_{30}} &
        0                                  &
        -f_{12} x_{10}
        \\
        %===
        0 &
        -\lr{g_2 - f_{23} x_{30}} &
        \lr{f_{23} x_{20}+ g_{23}}
        \\
        %===
        -f_{31} x_{30}  &
        g_{32} - f_{32} x_{30}&
        -f_{32} x_{20} - g_3 - f_{31} x_{10}
    }
    \bm{\delta x_1\\
        \delta x_2\\
        \delta x_3\\
        }
    + b \bm{\delta u_1 \\ \delta u_2 \\ 0}
\end{equation}




%\input{secs/kaushal_model.tex}

