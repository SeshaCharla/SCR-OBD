\section{Ammonia input (Urea Dosing) dynamics}
The actual input to the system is urea from AdBlue ($32.5\%$ aqueous urea
solution) injection that converted to ammonia (through reactions:
(\ref{eqn::urea_1}), (\ref{eqn::urea_2}) and (\ref{eqn::urea_3})). This can be modelled by the following eqation \cite{nova2014urea}:
\begin{align*}
    \dot C_{NH_3, in} &= - \frac{1}{\tau} C_{NH_3, in} + 2 \frac{1}{\tau} \frac{ \eta u_{Adblue}}{N_{urea} F}\\
    \text{where, } \quad &\\
    \tau &- \text{Time constant}\\
    u_{AdBlue} &- \text{Mass injection rate of the AdBlue solution}\\
    \eta &- \text{Mass fraction of urea in the solution}\\
    N_{urea} &- \text{Atomic number of urea}\\
    F &- \text{Exhaust flow rate of the catalyst } m^3/s
\end{align*}
The above model is re-parametrized as follows:
\begin{align*}
    \dot C_{NH_3, in} &= \frac{1}{\tau} \lr{ -C_{NH_3, in} +\eta_{urea} C_{NH_3, ideal}} \\
    \text{where, }\qquad &\\
    \eta_{urea} &- \text{urea} \rightarrow NH_3 \text{ conversion efficiency}\\
    \tau &- \text{urea} \rightarrow NH_3 \text{ time-constant}\\
    C_{NH_3, ideal} &- \text{ideal $C_{NH_3, in}$ if $\eta_{urea} = 1$ (constant parameter obtained through calibration)}
\end{align*}
In the present work, $C_{NH_3, in}$ is considered as the input instead of AdBlue
injection as its dynamics are completely decoupled from that of other states.
Further, it is observed that AdBlue is completely converted to Ammonia at the
very upstream part of the SCR catalyst \cite{hsieh2011development}.
