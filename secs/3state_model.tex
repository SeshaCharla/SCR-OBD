\section{3 state dynamic model}
\itbf{Assumptions }:

The following are the additional assumptions along with the assumptions of
4-state model that are used to arrive at the three-state dynamic model\cite{devarakonda2009model}:
\begin{enumerate}
    \item Only the standard SCR reaction is considered.
    \item All $NO_x$ in the exhaust gas is assumed to be $NO$.
    \begin{itemize}
        \item The commercially available $NO_x$ sensor (Horiba gas analyzer \cite{nova2014urea}) cannot differentiate between $NO$ and $NO_2$.
    \end{itemize}
    \item Slow SCR reaction is neglected.
    \begin{itemize}
        \item The flow rate of the exhaust would ensure that the not a significant concentration of tail-pipe exhaust components are due to the slow SCR reaction \cite{nova2014urea}.
    \end{itemize}
    \item Mass transfer is neglected. That means the chemical kinetics in the catalyst are reaction controlled.
    \begin{itemize}
        \item The standard SCR reaction rate is faster than the flow rate of the exhaust fluids.
    \end{itemize}
    \item Nitrogen selectivity for ammonia oxidation is $100\%$.
    \begin{itemize}
        \item This assumption is relaxed by including algebraic relationship between selectivity and the temperature (ASC model \cite{jain2023diagnostics}).
    \end{itemize}
    \item Reaction rates are assumed to be a function of the gas phase concentration of $NO_x$ and ammonia storage.
\end{enumerate}

\itbf{Dynamic Model Derivation}:

We have the mass (moles of reactants in/out) balance from the CSTR assumption:
\begin{align*}
    \bm{V \dot C_{NO}\\ V \dot C_{NH_3} \\ \dot M_{NH_3} } &=
    \bm{
        -R_1 - F C_{NO}\\
        -R_{4F} + R_{4R} - F C_{NH_3}\\
        R_{4F} - R_{4R} - R_1 - R_3
    } +
    F \bm{1 & 0 \\ 0 & 1 \\ 0 & 0} \bm{ C_{NO, in} \\ C_{NH_3, in}}\\
    % ===
    \text{Let, } b &= \frac{F}{V}\\
\end{align*}
\begin{equation} \label{eqn::mass_balance}
    %===
    \bm{\dot C_{NO} \\  \dot C_{NH_3} \\ \dot M_{NH_3} } = \frac{1}{V}
    \bm{
        -R_1 - F C_{NO}\\
        -R_{4F} + R_{4R} - F C_{NH_3}\\
        R_{4F} - R_{4R} - R_1 - R_3
    } +
    b \bm{1 & 0 \\ 0 & 1 \\ 0 & 0} \bm{ C_{NO, in} \\ C_{NH_3, in}}\\
    %===
\end{equation}

%===============================================================================
\subsection{Dynamic Model with storage ratio $(\theta)$ as state}
Let,
\begin{align*}
    \bm{x_1 \\ x_2 \\ x_3} = \bm{C_{NO} \\ C_{NH_3} \\ \theta} \qquad &
    \bm{u_1 \\ u_2 } = \bm{C_{NO, in} \\ C_{NH_3, in}}
\end{align*}
Rewriting eqn.~\ref{eqn::mass_balance}:
\begin{equation}\label{eqn::3_state_theta}
    \bm{\dot x_1 \\ \dot x_2 \\ \dot x_3} =
    \bm{[k_1 \Theta] x_1 x_3 - b x_1 \\
        -[k_{4F} \Theta] x_2 (1-x_3) + [k_{4R} V^{-1} \Theta] x_3 - b x_2 \\
        +[k_{4F}V] x_2 (1-x_3) - [k_{4R}] x_3 - [k_1 V] x_1 x_3 - [k_3 ] x_3} +
    b \bm{1 & 0 \\ 0 & 1 \\ 0 & 0} \bm{u_1 \\ u_2}\\
\end{equation}

The following parameters are defined for convenience, based on the coefficients
of product of states and individual states in each of the equations:
\begin{align*}
    \text{Coefficients of product of states:} &\qquad& \text{Coefficients of states:}\\
    \mat{   & x_1    & x_2      & x_3    \\
        x_1 &        &          & f_{13} \\
        x_2 &        &          & f_{23} \\
        x_3 & f_{31} & f_{32}   &}
    &\qquad &
    \mat{    & x_1    & x_2      & x_3    \\
        x_1  & g_1    &          &        \\
        x_2  &        & g_{2}    & g_{23} \\
        x_3  &        & g_{32}   & g_{3}}
\end{align*}
\begin{align*}
    \mat{
    \\f_{13} &=& k_1 \Theta
    \\f_{23} &=& k_{4F} \Theta
    \\f_{32} &=& k_{4F} V
    \\f_{31} &=& k_1 V
    }
    \qquad \qquad
    \mat{
    \\ g_1    &=& b
    \\ g_2    &=& b + k_{4F} \Theta
    \\ g_{3}  &=& [k_{4R}+k_3]
    \\ g_{23} &=& k_{4R} V^{-1} \Theta
    \\ g_{32} &=& k_{4F} V
    }
\end{align*}
\begin{equation}\label{eqn::parm_model_theta}
     \bm{\dot x_1 \\
        \dot x_2\\
        \dot x_3\\
        } =
    \bm{
        -f_{12} x_1 x_3
        -g_1 x_1
        \\
        %===
        -g_2 x_2
        + f_{23} x_2 x_3
        + g_{23} x_3
        \\
        %===
        -f_{32} x_2 x_3
        -g_3 x_3
        -f_{31} x_1 x_3
        + g_{32} x_2
    }
    + b \bm{u_1 \\ u_2 \\ 0}
\end{equation}

\itbf{Note}: Some $f_{\bullet}\,'s, g_{\bullet}\,'s$ are algebraically related.

\subsubsection{Small Perturbation model}
We have the small-perturbation model from eqn.~\ref{eqn::parm_model_theta}:
\begin{equation}\label{eqn::sml_ptrb_theta}
     \bm{\delta \dot x_1 \\
        \delta \dot x_2\\
        \delta \dot x_3\\
        } =
    \bm{
        -\lr{g_1 + f_{12} x_{30}} &
        0                                  &
        -f_{12} x_{10}
        \\
        %===
        0 &
        -\lr{g_2 - f_{23} x_{30}} &
        \lr{f_{23} x_{20}+ g_{23}}
        \\
        %===
        -f_{31} x_{30}  &
        g_{32} - f_{32} x_{30}&
        -f_{32} x_{20} - g_3 - f_{31} x_{10}
    }
    \bm{\delta x_1\\
        \delta x_2\\
        \delta x_3\\
        }
    + b \bm{\delta u_1 \\ \delta u_2 \\ 0}
\end{equation}


%===============================================================================

%===============================================================================
\subsection{Dynamic Model with molar storage ratio $(M_{NH_3})$ as state}
Let,
\begin{align*}
    \bm{x_1 \\ x_2 \\ x_3} = \bm{C_{NO} \\ C_{NH_3} \\ M_{NH_3}} \qquad &
    \bm{u_1 \\ u_2 } = \bm{C_{NO, in} \\ C_{NH_3, in}}
\end{align*}
Rewriting eqn.~\ref{eqn::mass_balance}:
\begin{equation}\label{eqn::3_state_M}
    \bm{\dot x_1 \\ \dot x_2 \\ \dot x_3} =
    \bm{k_1 x_1 x_3 - b x_1 \\
        -k_{4F}  x_2 (\Theta-x_3) + [k_{4R} V^{-1}] x_3 - b x_2 \\
        +[k_{4F}V] x_2 (\Theta-x_3) - [k_{4R}] x_3 - [k_1V] x_1 x_3 - k_3 x_3} +
    b \bm{1 & 0 \\ 0 & 1 \\ 0 & 0} \bm{u_1 \\ u_2}\\
\end{equation}

The following parameters are defined for convenience, based on the coefficients
of product of states and individual states in each of the equations:
\begin{align*}
    \text{Coefficients of product of states:} &\qquad& \text{Coefficients of states:}\\
    \mat{   & x_1    & x_2      & x_3    \\
        x_1 &        &          & f_{13} \\
        x_2 &        &          & f_{23} \\
        x_3 & f_{31} & f_{32}   &}
    &\qquad &
    \mat{    & x_1    & x_2      & x_3    \\
        x_1  & g_1    &          &        \\
        x_2  &        & g_{2}    & g_{23} \\
        x_3  &        & g_{32}   & g_{3}}
\end{align*}
\begin{align*}
    \mat{
    \\f_{13} &=& k_1
    \\f_{23} &=& k_{4F}
    \\f_{32} &=& k_{4F} \Theta
    \\f_{31} &=& k_1 V
    }
    \qquad \qquad
    \mat{
    \\ g_1    &=& b
    \\ g_2    &=& b + k_{4F} \Theta
    \\ g_{3}  &=& k_{4R}+k_3
    \\ g_{23} &=& k_{4R} V^{-1}
    \\ g_{32} &=& k_{4F} V \Theta
    }
\end{align*}
\begin{equation}\label{eqn::parm_model_M}
     \bm{\dot x_1 \\
        \dot x_2\\
        \dot x_3\\
        } =
    \bm{
        -f_{12} x_1 x_3
        -g_1 x_1
        \\
        %===
        -g_2 x_2
        + f_{23} x_2 x_3
        + g_{23} x_3
        \\
        %===
        -f_{32} x_2 x_3
        -g_3 x_3
        -f_{31} x_1 x_3
        + g_{32} x_2
    }
    + b \bm{u_1 \\ u_2 \\ 0}
\end{equation}

\itbf{Note}: Some $f_{\bullet}'s, g_{\bullet}'s$ are algebraically related.

\subsubsection{Small Perturbation model}
We have the small-perturbation model from eqn.~\ref{eqn::parm_model_M}:
\begin{equation}\label{eqn::sml_ptrb_M}
     \bm{\delta \dot x_1 \\
        \delta \dot x_2\\
        \delta \dot x_3\\
        } =
    \bm{
        -\lr{g_1 + f_{12} x_{30}} &
        0                                  &
        -f_{12} x_{10}
        \\
        %===
        0 &
        -\lr{g_2 - f_{23} x_{30}} &
        \lr{f_{23} x_{20}+ g_{23}}
        \\
        %===
        -f_{31} x_{30}  &
        g_{32} - f_{32} x_{30}&
        -f_{32} x_{20} - g_3 - f_{31} x_{10}
    }
    \bm{\delta x_1\\
        \delta x_2\\
        \delta x_3\\
        }
    + b \bm{\delta u_1 \\ \delta u_2 \\ 0}
\end{equation}


%===============================================================================
\subsection{Parametrization w.r.t the choice of $x_3$}

\begin{figure}[H]
 \begin{align*}
    \mat{
    \\f_{13} &=& k_1 \Theta
    \\f_{23} &=& k_{4F} \Theta
    \\f_{32} &=& k_{4F} V
    \\f_{31} &=& k_1 V
    }
    \qquad \qquad
    \mat{
    \\ g_1    &=& b
    \\ g_2    &=& b + k_{4F} \Theta
    \\ g_{3}  &=& [k_{4R}+k_3]
    \\ g_{23} &=& k_{4R} V^{-1} \Theta
    \\ g_{32} &=& k_{4F} V
    }
\end{align*}
 \caption*{Parametrization if $x_3 = \theta_{NH_3}$}
\end{figure}
\begin{figure}[H]
\begin{align*}
    \mat{
    \\f_{13} &=& k_1
    \\f_{23} &=& k_{4F}
    \\f_{32} &=& k_{4F} \Theta
    \\f_{31} &=& k_1 V
    }
    \qquad \qquad
    \mat{
    \\ g_1    &=& b
    \\ g_2    &=& b + k_{4F} \Theta
    \\ g_{3}  &=& k_{4R}+k_3
    \\ g_{23} &=& k_{4R} V^{-1}
    \\ g_{32} &=& k_{4F} V \Theta
    }
\end{align*}
\caption*{Parametrization if $x_3 = M_{NH_3}$}
\end{figure}

In the second parametrization $\Theta$ is only coefficient of $k_{4F}$. This can
reduce the variance of estimation.
