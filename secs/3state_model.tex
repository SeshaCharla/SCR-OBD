\subsection{3 state dynamic model}
\subsubsection{Assumptions }
The following are the additional assumptions along with the assumptions of
4-state model that are used to arrive at the three-state dynamic model\cite{devarakonda2009model}:
\begin{enumerate}
    \item Only the standard SCR reaction is considered.
    \item All $NO_x$ in the exhaust gas is assumed to be $NO$.
    \begin{itemize}
        \item The commercially available $NO_x$ sensor (Horiba gas analyzer \cite{nova2014urea}) cannot differentiate between $NO$ and $NO_2$.
    \end{itemize}
\end{enumerate}
The above two assumptions result in the following:
\begin{align*}
    C_{NO_2} &= 0 \qquad C_{NO_2,  in} = 0\\
    r_5 &= 0
\end{align*}

\subsubsection{Dynamic Model}
Including the above assumptions in the 4-state dynamic model
(\ref{eqn::4_state}):

\begin{equation} \label{eqn::3_state}
    \bm{\dot C_{NO} \\
        \dot C_{NH_3}\\
        \dot \theta_{NH_3}\\
        } =
    \bm{
        -r_1 C_{NO} C_{O_2} \theta_{NH_3} \Theta V
        -b C_{NO}\\
        %===
        -C_{NH_3} \lrb{\Theta r_{rF} \lr{1 - \theta_{NH_3}} + b} + V^{-1} r_{4R} \Theta \theta_{NH_3}\\
        %===
        -\theta_{NH_3} \lr{r_{4F}C_{NH_3} V + r_3 C_{O_2}V + r_{4R} + r_1 C_{NO} C_{O_2} V^2 } + r_{4F} C_{NH_3} V
    }
    + b \bm{C_{NO, in} \\ C_{NH_3, in} \\ 0}
\end{equation}

The following parameters are defined for convenience:
\begin{align*}
    \mat{
    p_1 = r_1 C_{O_2} V
    &p_2 =  \frac{r_{4R}}{V}
    &p_3 = r_{4F} V
    &p_4 = r_3 C_{O_2} V
    &p_5 = r_1 C_{O_2} V^2
    }
\end{align*}
\begin{align*}
     \bm{\dot C_{NO} \\
        \dot C_{NH_3}\\
        \dot \theta_{NH_3}\\
        } &=
    \bm{
        -p_1 C_{NO} \theta_{NH_3} \Theta
        -b C_{NO}
        \\
        %===
        -C_{NH_3} \lr{\Theta r_{4F} + b}
        + r_{4F} \Theta C_{NH_3} \theta_{NH_3}
        + p_2 \Theta \theta_{NH_3}
        \\
        %===
        -p_3C_{NH_3} \theta_{NH_3}
        -\lr{p_4 + r_{4R}} \theta_{NH_3}
        -p_5 C_{NO} \theta_{NH_3}
        + p_3 C_{NH_3}
    }
    + b \bm{C_{NO, in} \\ C_{NH_3, in} \\ 0}
\end{align*}
Defining the following coefficients of product of states and states:
\begin{align*}
    \text{Coefficients of product of states:} &\qquad& \text{Coefficients of states:}\\
    \mat{             & C_{NO} & C_{NH_3} & \theta_{NH_3} \\
        C_{NO}        &        & f_{12}   &        \\
        C_{NH_3}      &        &          & f_{23} \\
        \theta_{NH_3} & f_{31} & f_{32}   &}
    &\qquad &
    \mat{             & C_{NO} & C_{NH_3} & \theta_{NH_3} \\
        C_{NO}        & g_1    &          &        \\
        C_{NH_3}      &        & g_{2}    & g_{23}        \\
        \theta_{NH_3} &        & g_{32}   & g_{3}}
\end{align*}
\begin{align*}
    \mat{
    \\f_{12} &=& p_1 \Theta    &=& r_1 C_{O_2} V \Theta
    \\f_{23} &=& r_{4F} \Theta & &
    \\f_{32} &=& p_3           &=& r_{4F} V
    \\f_{31} &=& p_5           &=& r_1 C_{O_2} V^2}
    \qquad \qquad
    \mat{
    \\ g_1    &=& b                      & &
    \\ g_2    &=& \lr{\Theta r_{4F} + b} & &
    \\ g_{3}  &=& p_4 + r_{4R}           &=& r_3 C_{O_2} V + r_{4R}
    \\ g_{23} &=& p_2 \Theta             &=&  \frac{r_{4R}}{V} \Theta
    \\ g_{32} &=& p_3                    &=& r_{4F} V
    }
\end{align*}
\begin{equation*}
     \bm{\dot C_{NO} \\
        \dot C_{NH_3}\\
        \dot \theta_{NH_3}\\
        } =
    \bm{
        -f_{12} C_{NO} \theta_{NH_3}
        -g_1 C_{NO}
        \\
        %===
        -g_2 C_{NH_3}
        + f_{23} C_{NH_3} \theta_{NH_3}
        + g_{23} \theta_{NH_3}
        \\
        %===
        -f_{32} C_{NH_3} \theta_{NH_3}
        -g_3 \theta_{NH_3}
        -f_{31} C_{NO} \theta_{NH_3}
        + g_{32} C_{NH_3}
    }
    + b \bm{C_{NO, in} \\ C_{NH_3, in} \\ 0}
\end{equation*}
Let,
\begin{align*}
    x_1 &= C_{NO} &  u_1 &= C_{NO, in}\\
    x_2 &= C_{NH_3} & u_2 &= C_{NH_3, in}\\
    x_3 &= \theta_{NH_3} & &
\end{align*}
\begin{equation}\label{eqn::parm_model}
     \bm{\dot x_1 \\
        \dot x_2\\
        \dot x_3\\
        } =
    \bm{
        -f_{12} x_1 x_3
        -g_1 x_1
        \\
        %===
        -g_2 x_2
        + f_{23} x_2 x_3
        + g_{23} x_3
        \\
        %===
        -f_{32} x_2 x_3
        -g_3 x_3
        -f_{31} x_1 x_3
        + g_{32} x_2
    }
    + b \bm{u_1 \\ u_2 \\ 0}
\end{equation}


\subsubsection{Small Perturbation model}
We have the small-perturbation model from eqn.~\ref{eqn::parm_model}:
\begin{equation}\label{eqn::sml_ptrb}
     \bm{\delta \dot x_1 \\
        \delta \dot x_2\\
        \delta \dot x_3\\
        } =
    \bm{
        -\lr{g_1 + f_{12} x_{30}} &
        0                                  &
        -f_{12} x_{10}
        \\
        %===
        0 &
        -\lr{g_2 - f_{23} x_{30}} &
        \lr{f_{23} x_{20}+ g_{23}}
        \\
        %===
        -f_{31} x_{30}  &
        g_{32} - f_{32} x_{30}&
        -f_{32} x_{20} - g_3 - f_{31} x_{10}
    }
    \bm{\delta x_1\\
        \delta x_2\\
        \delta x_3\\
        }
    + b \bm{\delta u_1 \\ \delta u_2 \\ 0}
\end{equation}




%% ==============================================================================
\newpage
\subsubsection{Comparing with previous dynamic model }
The 3-state dynamic model used previously \cite{jain2023diagnostics} is as follows:
\begin{figure}[H]
    \begin{minipage}{0.59\textwidth}
        \begin{align*}
        &\text{\itbf{Dynamics}:}\\
            \dot x_1 &= \lr{\frac{F}{V}} \lr{u_1 - x_1} - K \alpha_{ads} x_1 (1 - x_3) + K \alpha_{des} x_3\\
            %===
            \dot x_2 &= \lr{\frac{F}{V}} \lr{u_2 - x_2} - K \alpha_{SCR} x_2 x_3\\
            %===
            \dot x_3 &= \alpha_{ads} x_1 \lr{ 1 - x_3} - \alpha_{SCR} x_2 x_3 - \alpha_{des} x_3 - \alpha_{oxi} x_3
        \end{align*}
    \end{minipage}
    \begin{minipage}{0.4\textwidth}
        \begin{align*}
            &\\
            \dot u_1 &= \frac{1}{\tau} \lr{ -u_1 +\eta_{urea} u_{1, ideal}} & [\text{actuator dynamics}]\\
            %===
            K &= \frac{S_1}{V} \exp\lr{-S_2 T}\\
            %===
            \alpha_i &= A_i \exp\lr{-\frac{E_i}{RT}}
            %===
            & i = ads, des, SCR, oxi
        \end{align*}
    \end{minipage}
    \begin{minipage}{\textwidth}
        \begin{align*}
            \text{\itbf{States}:}&\\
            %===
            x_1 &- NH_3 \text{ concentration at SCR-out } \lr{mol/m^3}\\
            %===
            x_2 &- NO_x \, (NO) \text{ concentration at SCR-out } \lr{mol/m^3}\\
            %===
            x_3 &- NH_3 \text{ storage capacity fraction in SCR } = \frac{\text{Moles of $NH_3$ adsorbed}}{\text{Total moles of $NH_3$ that can be adsorbed}}\\
            %===
            \text{\itbf{Inputs}:}&\\
            %===
            u_1 &- NH_3 \text{ injected concentration at SCR-in } \lr{mol/m^3}\\
            &\qquad \begin{matrix*}[l]
                \eta_{urea} &-& \text{urea} \rightarrow NH_3 \text{ conversion efficiency}\\
                 \tau &-& \text{urea} \rightarrow NH_3 \text{ time-constant}\\
                 u_{1, ideal} &-& \text{ideal $u_1$ if $\eta_{urea} = 1$ (constant parameter obtained through calibration)}\\
            \end{matrix*}\\
            %===
            u_2 &- NO_x (NO) \text{ concentration at SCR-in } \lr{mol/m^3}\\
            %===
            \text{\itbf{Parameters}:}&\\
            K &- \text{SCR catalyst's } NH_3 \text{ storage capacity } \lr{mols}\\
            &\qquad \begin{matrix*}[l]
                S_1, S_2 &-& \text{Aging parameters of the catalyst}
            \end{matrix*}\\
            F &- \text{Exhaust gas volume flow rate } \lr{m^3/s}\\
            V &- \text{Volume of the catalyst } \lr{m^3}\\
            \alpha_i &- \text{Reaction rate of $i^{th}$ reaction}\\
            &\qquad \begin{matrix*}[l]
            E_i &-& \text{Activation Energy of $i^{th}$ reaction}\\
            A_i &-& \text{Pre-exponential factor of $i^{th}$ reaction}\\
            i : SCR &-& \text{Standard SCR reaction}\\
            i : ads &-& NH_3 \text{ adsorption}\\
            i : des &-& NH_3 \text{ desorption}\\
            i : oxi &-& NH_3 \text{ oxidation}
            \end{matrix*}
                   \end{align*}
    \end{minipage}
\end{figure}

